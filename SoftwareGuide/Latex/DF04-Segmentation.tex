
\chapter{Segmentation}

Segmentation of medical images is a challenging task. A myriad of different
methods have been proposed and implemented in recent years. In spite of the
huge effort invested in this problem, there is no single approach that can
generally solve the problem of segmentation for the large variety of image
modalities existing today.

The most effective segmentation algorithms are obtained by carefully
customizing combinations of components. The parameters of these components are
tuned for the characteristics of the image modality used as input and the
features of the anatomical structure to be segmented.

The Insight Toolkit provides a basic set of algorithms that can be used to
develop and customize a full segmentation application. Some of the most
commonly used segmentation components are described in the following
sections.


\section{Region Growing}

Region growing algorithms have proven to be an effective approach for image
segmentation. The basic approach of a region growing algorithm is to start
from a seed region (typically one or more pixels) which is considered to be
inside the object to be segmented. The pixels neighboring this region are
evaluated to determine if they should also be considered part of the
object. If so, they are added to the region and the process continues as long
as new pixels are added to the region.  Region growing algorithms vary
depending on the criteria used to decide whether a pixel should be included
in the region or not, the type connectivity used to determine neighbors, and
the strategy used to visit neighboring pixels.

Several implementations of region growing are available in ITK.  This section
describes some of the most commonly used algorithms.

\subsection{Connected Threshold}

A simple criterion for including pixels in a growing region is to evaluate
intensity value inside a specific interval.

\label{sec:ConnectedThreshold}
\ifitkFullVersion
\input{ConnectedThresholdImageFilter.tex}
\fi

\subsection{Otsu Segmentation}
Another criterion for classifying pixels is to minimize the error of misclassification.
The goal is to find a threshold that classifies the image into two clusters such that
we minimize the area under the histogram for one cluster that lies on the other cluster's
side of the threshold. This is equivalent to minimizing the within-class variance
or equivalently maximizing the between-class variance.

\label{sec:OtsuThreshold}
\ifitkFullVersion
\input{OtsuThresholdImageFilter.tex}
\fi

\label{sec:OtsuMultipleThreshold}
\ifitkFullVersion
\input{OtsuMultipleThresholdImageFilter.tex}
\fi

\subsection{Neighborhood Connected}
\label{sec:NeighborhoodConnectedImageFilter}
\ifitkFullVersion
\input{NeighborhoodConnectedImageFilter.tex}
\fi


\subsection{Confidence Connected}
\label{sec:ConfidenceConnected}
\ifitkFullVersion
\input{ConfidenceConnected.tex}
\subsubsection{Application of the Confidence Connected filter on the Brain Web Data}
This section shows some results obtained by applying the Confidence Connected filter on the BrainWeb database. The filter was applied on a 181 $\times$ 217 $\times$ 181 crosssection of the {\it brainweb165a10f17} dataset. The data is a MR T1 acquisition, with an intensity non-uniformity of 20\% and a slice thickness 1mm. The dataset may be obtained from
\code{http://www.bic.mni.mcgill.ca/brainweb/} or
\code{ftp://public.kitware.com/pub/itk/Data/BrainWeb/}.

The previous code was used in this example replacing the image dimension by 3.
Gradient Anistropic diffusion was applied to smooth the image. The filter used 2 iterations, a time step of 0.05 and a conductance value of 3. The smoothed volume was then segmented using the Confidence Connected approach. Five seed points were used at coordinate locations (118,85,92), (63,87,94), (63,157,90), (111,188,90), (111,50,88). The ConfidenceConnnected filter used the parameters, a neighborhood radius of 2, 5 iterations and an $f$ of 2.5 (the same as in the previous example). The results were then rendered using VolView.

Figure \ref{fig:3DregionGrowingScreenshot1} shows the rendered volume. Figure \ref{fig:SlicesBrainWeb} shows an axial, saggital and a coronal slice of the volume.

\begin{figure}
\center
\includegraphics[width=0.6\textwidth]{3DregionGrowingScreenshot1.eps}
\itkcaption[Whitematter Confidence Connected segmentation.]{White matter segmented using Confidence Connected region growing.}
\label{fig:3DregionGrowingScreenshot1}
\end{figure}

\begin{figure}
\center
\includegraphics[width=\textwidth]{SlicesBrainWebConfidenceConnected.eps}
\itkcaption[Axial, sagittal, and coronal slice of Confidence Connected segmentation.]{Axial, sagittal and coronal slice segmented using Confidence Connected region growing.}
\label{fig:SlicesBrainWeb}
\end{figure}



\fi




\subsection{Isolated Connected}
\label{sec:IsolatedConnected}
\ifitkFullVersion
\input{IsolatedConnectedImageFilter.tex}
\fi


\subsection{Confidence Connected in Vector Images}
\label{sec:VectorConfidenceConnected}
\ifitkFullVersion
\input{VectorConfidenceConnected.tex}
\fi


\section{Segmentation Based on Watersheds}
\label{sec:WatershedSegmentation}
\ifitkFullVersion
\input WatershedSegmentation.tex
\fi


% the clearpage command helps to avoid orphans in the title of the next
% section.
\clearpage

\section{Level Set Segmentation}
\label{sec:LevelSetsSegmentation}
\ifitkFullVersion
%%%%%%%%%%%%%%%%%%%%%%%%%%%%%%%%%%%%%%%%%%%%%%%%%%%%%%%%%%%%%%%%%%%%%%%%
%
%
%     This file is included from the file   Segmentation.tex
%
%     Section tag and label are placed in this top file.
%
%
%
%%%%%%%%%%%%%%%%%%%%%%%%%%%%%%%%%%%%%%%%%%%%%%%%%%%%%%%%%%%%%%%%%%%%%%%%



\begin{floatingfigure}[rlp]{0.5\textwidth}
  \centering
  \includegraphics[width=6cm]{LevelSetZeroSet.eps}
  \caption[Zero Set Concept]{Concept of zero set in a level set.\label{fig:LevelSetZeroSet}}
\end{floatingfigure}

The paradigm of the level set is that it is a numerical method for tracking
the evolution of contours and surfaces. Instead of manipulating the contour
directly, the contour is embedded as the zero level set of a higher
dimensional function called the level-set function, $\psi(\bf{X},t)$. The
level-set function is then evolved under the control of a differential
equation.  At any time, the evolving contour can be obtained by extracting
the zero level-set $\Gamma(\bf{X},t) =
\{\psi(\bf{X},t) = 0\}$ from the output.  The main advantages of using level
sets is that arbitrarily complex shapes can be modeled and topological
changes such as merging and splitting are handled implicitly.

Level sets can be used for image segmentation by using image-based features
such as mean intensity, gradient and edges in the governing differential
equation.  In a typical approach, a contour is initialized by a user and is
then evolved until it fits the form of an anatomical structure in the image.
Many different implementations and variants of this basic concept have been
published in the literature. An overview of the field has been made by
Sethian \cite{Sethian1996}.

The following sections introduce practical examples of some
of the level set segmentation methods available in ITK.  The remainder of this
section describes features common to all of these filters except the
\doxygen{FastMarchingImageFilter}, which is derived from a different code
framework.  Understanding these features will aid in using the filters
more effectively.

Each filter makes use of a generic level-set equation to compute the update to
the solution $\psi$ of the partial differential equation.

\begin{equation}
\label{eqn:LevelSetEquation}
\frac{d}{dt}\psi = -\alpha \mathbf{A}(\mathbf{x})\cdot\nabla\psi - \beta
  P(\mathbf{x})\mid\nabla\psi\mid +
\gamma Z(\mathbf{x})\kappa\mid\nabla\psi\mid
\end{equation}

where $\mathbf{A}$ is an advection term, $P$ is a propagation (expansion) term,
and $Z$ is a spatial modifier term for the mean curvature $\kappa$.  The scalar
constants $\alpha$, $\beta$, and $\gamma$ weight the relative influence of
each of the terms on the movement of the interface.  A segmentation filter may
use all of these terms in its calculations, or it may omit one or more terms.
If a term is left out of the equation, then setting the corresponding scalar
constant weighting will have no effect.

All of the level-set based segmentation filters \emph{must} operate with
floating point precision to produce valid results.  The third, optional
template parameter is the \emph{numerical type} used for calculations and as
the output image pixel type.  The numerical type is \code{float} by default,
but can be changed to \code{double} for extra precision.  A user-defined,
signed floating point type that defines all of the necessary arithmetic
operators and has sufficient precision is also a valid choice.  You should
not use types such as \code{int} or \code{unsigned char} for the numerical
parameter.  If the input image pixel types do not match the numerical type,
those inputs will be cast to an image of appropriate type when the filter is
executed.

Most filters require two images as input, an initial model $\psi(\bf{X},
t=0)$, and a \emph{feature image}, which is either the image you wish to
segment or some preprocessed version.  You must specify the isovalue that
represents the surface $\Gamma$ in your initial model. The single image
output of each filter is the function $\psi$ at the final time step.  It is
important to note that the contour representing the surface $\Gamma$ is the
zero level-set of the output image, and not the isovalue you specified for
the initial model.  To represent $\Gamma$ using the original isovalue, simply
add that value back to the output.

The solution $\Gamma$ is calculated to subpixel precision.  The best discrete
approximation of the surface is therefore the set of grid positions closest to
the zero-crossings in the image, as shown in
Figure~\ref{fig:LevelSetSegmentationFigure1}.  The
\doxygen{ZeroCrossingImageFilter} operates by finding exactly those grid
positions and can be used to extract the surface.

\begin{figure}
\centering
\includegraphics[width=0.4\textwidth]{LevelSetSegmentationFigure1.eps}
\itkcaption[Grid position of the embedded level-set surface.]{The implicit level
set surface $\Gamma$ is the black line superimposed over the image grid.  The location
of the surface is interpolated by the image pixel values.  The grid pixels
closest to the implicit surface are shown in gray. }
\protect\label{fig:LevelSetSegmentationFigure1}
\end{figure}

There are two important considerations when analyzing the processing time for
any particular level-set segmentation task: the surface area of the evolving
interface and the total distance that the surface must travel.  Because the
level-set equations are usually solved only at pixels near the surface (fast
marching methods are an exception), the time taken at each iteration depends on
the number of points on the surface.  This means that as the surface grows, the
solver will slow down proportionally.  Because the surface must evolve slowly
to prevent numerical instabilities in the solution, the distance the surface
must travel in the image dictates the total number of iterations required.

Some level-set techniques are relatively insensitive to initial conditions
and are therefore suitable for region-growing segmentation. Other techniques,
such as the \doxygen{LaplacianSegmentationLevelSetImageFilter}, can easily
become ``stuck'' on image features close to their initialization and should
be used only when a reasonable prior segmentation is available as the
initialization.  For best efficiency, your initial model of the surface
should be the best guess possible for the solution.  When extending the
example applications given here to higher dimensional images, for example,
you can improve results and dramatically decrease processing time by using a
multi-scale approach. Start with a downsampled volume and work back to the
full resolution using the results at each intermediate scale as the
initialization for the next scale.


\subsection{Fast Marching Segmentation}
\label{sec:FastMarchingImageFilter}

\ifitkFullVersion
\input{FastMarchingImageFilter.tex}
\fi


\subsection{Shape Detection Segmentation}
\label{sec:ShapeDetectionLevelSetFilter}

\ifitkFullVersion
\input{ShapeDetectionLevelSetFilter.tex}
\fi


\subsection{Geodesic Active Contours Segmentation}
\label{sec:GeodesicActiveContourImageFilter}

\ifitkFullVersion
\input{GeodesicActiveContourImageFilter.tex}
\fi


\subsection{Threshold Level Set Segmentation}
\label{sec:ThresholdSegmentationLevelSetImageFilter}
\ifitkFullVersion
\input{ThresholdSegmentationLevelSetImageFilter.tex}
\fi


\subsection{Canny-Edge Level Set Segmentation}
\label{sec:CannySegmentationLevelSetImageFilter}
\ifitkFullVersion
\input{CannySegmentationLevelSetImageFilter.tex}
\fi


\subsection{Laplacian Level Set Segmentation}
\label{sec:LaplacianSegmentationLevelSetImageFilter}
\ifitkFullVersion
\input{LaplacianSegmentationLevelSetImageFilter.tex}
\fi

\subsection{Geodesic Active Contours Segmentation With Shape Guidance}
\label{sec:GeodesicActiveContourShapePriorLevelSetImageFilter}
\ifitkFullVersion
\input{GeodesicActiveContourShapePriorLevelSetImageFilter.tex}
\fi



\fi


%HACK: TODO Review what happended to these.
%REMOVED PATENTED: \section{Hybrid Methods}
%REMOVED PATENTED: \label{sec:HybridSegmentationMethods}
%REMOVED PATENTED:
%REMOVED PATENTED: \ifitkFullVersion
%REMOVED PATENTED: \input{HybridSegmentationMethods.tex}
%REMOVED PATENTED: \fi


\section{Feature Extraction}
\label{sec:FeatureExtractionMethods}

\ifitkFullVersion
\input{FeatureExtractionMethods.tex}
\fi
